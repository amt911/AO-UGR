% TODO: ARREGLAR JERARQUIA DE LAS MANOS

\section{Cambio de mano}

Antes de nada, voy a explicar la composición de las manos y de la espada en la escena:

\begin{itemize}
    \item \textbf{Manos: }Están formadas por una esfera, que simula las manos, y un cilindro, que simula el antebrazo. Cabe destacar que para facilitar su animación, he realizado una jerarquía en la que las manos son el padre de los antebrazos.
    
    % foto de eso
    \item \textbf{Espada: }La espada la he realizado utilizando como empuñadura un cilindro, un cubo achatado como guarda y para la hoja una pirámide de base rectangular y estirado en el eje Z para que sea más alto y puntiagudo.
    
    \bigskip

    Todas estas piezas las he agrupado para que sea más fácil trabajar con ellas. No obstante, al agrupar las piezas, el pivote se ha movido a la hoja, haciendo que cuando la agarre la mano no sea realista. Por tanto, es necesario mover el pivote de la espada de nuevo a la empuñadura.
    
    % foto de la espada
    % caption{Forma final de la espada.}
\end{itemize}

Voy a dividir la configuración de las manos y de la espada en distintas subsecciones.

\subsection{Configuración de las manos}

Para la animación del cambio de mano, he utilizado los siguientes \textit{keyframes} para la mano más a la izquierda de la escena:

\begin{itemize}
    \item \textbf{Instante 0: }La mano se encuentra en su posición inicial.
    \item \textbf{Instante : }La mano se ha acercado a la otra para darle la espada.
    % \item \textbf{Instante : }
    % \item \textbf{Instante : }
    % \item \textbf{Instante : }
    % \item \textbf{Instante : }
\end{itemize}

Las curvas de la animación son:

% curvas

Como se puede ver, en todas las curvas he usado la forma por defecto \textit{Slow-in/Slow-out}, ya que genera unos resultados más orgánicos y acordes a estas extremidades.


Mientras que la animación para la mano más a la derecha de la escena es: 

\begin{itemize}
    \item \textbf{Instante : }La mano se encuentra en su posición inicial y ha recibido la espada de la otra mano.
    \item \textbf{Instante : }Ahora la mano se ha dirigido a la plataforma de abajo de la grúa para dejar la espada.
    \item \textbf{Instante : }Finalmente la mano vuelve a la posición inicial.
\end{itemize}

Las curvas de animación para esta mano son:

% curvas

De nuevo, se peude ver que en todas las curvas se ha utilizado la misma forma que para la otra mano, para dar un resultado más realista. 

\bigskip

Además, se puede observar como hay una animación en el eje Z. Esto es debido a que la plataforma se encuentra ligeramente más alta que los brazos, haciendo que este tenga que subir un poco y luego bajar para llegar a su posición inicial.

\bigskip

En cuanto a restricciones, ambas manos no tienen restricción de ningún tipo, es la espada la que tiene restricciones.


\subsection{Espada}

% HABLAR SOLO DE LA RESTRICCION DE POSICION, LA OTRA VA EN EL COCHE

La espada tiene \dots


\subsection{Resultado final en el cambio de manos}

El resultado final de todo esto es el siguiente:

% fotos de los fotogramas mas importantes.