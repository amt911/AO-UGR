\section{Cámara}

Para la cámara he utilizado una de tipo \textit{Target}, cuya posición comienza en la izquierda de la escena.

Voy a dividir en dos subsecciones la animación del propio movimiento de la cámara y del seguimiento de los objetos de la escena:

% REVISAR ESTA PARTE PORQUE PUEDE SER QUE HAGA FALTA USAR UN PATH
\subsection{Movimiento de la cámara}

Para realizar el movimiento de pasar de izquierda a la derecha de la escena lo he animado de forma manual usando los siguientes \textit{keyframes}:

\begin{itemize}
    \item \textbf{Instante : }La cámara se encuentra a la izquierda de la escena.
    \item \textbf{Instante : }La cámara se encuentra a la derecha de la escena.
\end{itemize}

\bigskip

Las curvas de animación son las siguientes: 

% curvas de animacion
\begin{figure}[H]
    \centering
   \includegraphics[width=0.5\textwidth]{example-image-a}
\end{figure}
\begin{figure}[H]
    \centering
   \includegraphics[width=0.5\textwidth]{example-image-b}
\end{figure}

% explicacion de las curvas
\blindtext

\subsection{Seguimiento de los objetos}

Para realizar el seguimiento de los objetos he utilizado en el \textit{target} de la cámara una restricción de posición (\textit{Position Constraint}), cuyos objetivos son la espada y el coche.

\bigskip

Estos pesos deben cambiar cuando el coche proceda a moverse, para pasar de fijarse en la espada al propio coche. Los \textit{keyframes} de esta restricción son:

\begin{itemize}
    \item \textbf{Instante : }El \textit{target} está fijado completamente en la espada y la está siguiendo.
    \item \textbf{Instante : }El \textit{target} ahora está fijándose completamente en el coche y lo sigue.
\end{itemize}

\bigskip

En cuanto a las curvas de animación utilizadas, son las siguientes:

\begin{figure}[H]
    \centering
   \includegraphics[width=0.5\textwidth]{example-image-a}
\end{figure}
\begin{figure}[H]
    \centering
   \includegraphics[width=0.5\textwidth]{example-image-b}
\end{figure}

% explicacion
\blindtext