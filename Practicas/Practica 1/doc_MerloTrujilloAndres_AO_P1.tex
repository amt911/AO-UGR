\documentclass{article}
\usepackage[utf8]{inputenc}
\usepackage[spanish]{babel}
\usepackage{graphicx, graphics, float, fancyhdr, titling}
\usepackage{listings}
\usepackage[a4paper, total={6in, 9.5in}]{geometry}
\usepackage{fancyhdr}
\usepackage{hyperref}   %para que funcione addcontentsline debe ser la ultima que se cargue

%\setcounter{secnumdepth}{-2}       %Poner solo esto si no se quieren numero delante de las secciones y niveles inferiores.

\renewcommand{\footrulewidth}{0.4pt}
\title{
\includegraphics[width=1.75in]{imagenes/UGR-Logo.png} \\
\vspace*{1in}
\textbf{Memoria de la práctica 1} \\
Animación por Ordenador \\
\vspace*{0.5in}}
\author{Andrés Merlo Trujillo \\
andresmerlo@correo.ugr.es \\
77147239H \\ 
\vspace*{0.5in} \\
E.T.S. de Ingenierías Informática y de Telecomunicación \\
\textbf{Universidad de Granada}} \date{\today}

\hypersetup{
    colorlinks=true,
    linkcolor=black,
}

\renewcommand\maketitlehooka{\null\mbox{}\vfill}
\renewcommand\maketitlehookd{\vfill\null}

\begin{document}
\begin{titlingpage}
\maketitle
\end{titlingpage}

\tableofcontents

\newpage

\pagestyle{fancy}   %a partir de comienza el header (se salta el indice y portada)
\fancyhead[L]{Andrés Merlo Trujillo}
\fancyhead[R]{Animación por Ordenador}
%\section{Ejercicio 1}
%\begin{figure}[H]
%    \centering
%    \includegraphics[width=\textwidth]{imagenes/passwdfile.png}
%\end{figure}

\section{Ejercicio 1}

En este ejercicio se pide realizar dos animaciones: una primera de una pelota que rebota sobre un cubo, y una segunda con una pelota que empuja un cubo.

\bigskip


Lo primero que he realizado ha sido crear dos cubos y dos esfera y los he colocado en el sitio inicial; es decir, la primera en el aire y la otra lejos, aún sin tocar el cubo.


%imagen de la escena inicial.


En cuanto a la animacion, lo voy a dividir en dos subsecciones.

\subsection{Pelota que rebota sobre cubo (P1, C1)}

En este caso, solo es necesario animar la pelota, ya que el cubo se va a mantener inmovil a lo largo de toda la animacion. Tambien he querido hacerlo un poco mas realista, por lo que he animado mediante \textit{keyframes} un intento de aceleracion de la gravedad, pero el resultado no es muy convincente. Para ello, he tenido en cuenta que en cada rebote, la altura de la pelota va a ser cada vez menor y que va a tardar algo menos en caer de nuevo. A modo de resumen, la pelota tiene los siguientes \textit{keyframes}:

\begin{enumerate}
    \item Inicial que se encuentra en el aire.
    \item Sobre el cubo tocandolo (porque ha caido).
    \item En el aire de nuevo, pero mas bajo.
    \item Sobre el cubo de nuevo.
    \item Se repiten los dos pasos anteriores varias veces, cada vez con menos altura.
\end{enumerate}

A continuacion se encuentran las imagenes de todos los \textit{keyframes} de esta animacion:

%imagenes


Cabe destacar que si bien no se pedia modificar la curva, lo he realizado para que tenga aun mas realismo la animacion (dentro de lo que cabe). A continuacion muestro una captura de la forma de la curva de la animacion:

%imatgen de la curva


Como se puede observar, cuando la pelota se encuentra en lo alto, comeinza parada y poco a poco va acelerando debido a al gravedad, y cuando rebota en el cubo, mantiene la velocidad y poco a poco va frenando de nuevo por la gravedad, hasta llegar al punto mas alto, que cae de nuevo siguiendo los pasos anteriores.

\subsection{Pelota que empuja cubo (P2, C2)}

mec

\end{document}
