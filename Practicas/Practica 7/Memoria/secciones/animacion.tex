\section{Animación}

Antes de nada, la animación la he realizado mediante los \textit{sliders}, ya que permiten tener un control preciso de cada pieza y ayuda a no salirse de las restricciones de movimiento impuestas.

\bigskip

La animación configurada consiste en que la excavadora mueve el brazo al suelo para recoger con la pala tierra. Después, mueve toda la cabina hacia la izquierda para tirarla en otro lado. Finalmente, la excavadora gira de nuevo, estando de nuevo en la posición cero.

\bigskip

Los \textit{keyframes} de la animación son:

\begin{itemize}
    \item \textbf{Instante 0: }La excavadora está en su posición inicial.
    \item \item \textbf{Instante 0: }
    \item \item \textbf{Instante 0: }
    \item \item \textbf{Instante 0: }
    \item \item \textbf{Instante 0: }
    \item \item \textbf{Instante 0: }
    \item \item \textbf{Instante 0: }
\end{itemize}

% fotos de los keyframes mas importantes


Mientras que las curvas de animación son: 

% foto de las curvas de aniamción

He decidido utilizar el tipo de curva \textit{Slow-in/Slow-out}, ya que es el que me ha parecido más realista de todos, ya que el brazo necesita unos instantes para llegar a la velocidad y otros instantes para parar.