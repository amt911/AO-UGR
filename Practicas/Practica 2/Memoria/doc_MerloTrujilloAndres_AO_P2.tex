%TODO: 
% ARREGLAR FUNCION EXPONENCIAL CRECIENTE Y DECRECIENTE, DEPENDE DEL CONTEXTO
% MULETILLAS: ADEMAS, CABE DESTACAR
\documentclass{article}
\usepackage[utf8]{inputenc}
\usepackage[spanish]{babel}
\usepackage{graphicx, graphics, float, fancyhdr, titling}
\usepackage{listings}
\usepackage[a4paper, total={6in, 9.5in}]{geometry}
\usepackage{fancyhdr}
\usepackage{hyperref}   %para que funcione addcontentsline debe ser la ultima que se cargue

%\setcounter{secnumdepth}{-2}       %Poner solo esto si no se quieren numero delante de las secciones y niveles inferiores.

\renewcommand{\footrulewidth}{0.4pt}
\title{
\includegraphics[width=1.75in]{imagenes/UGR-Logo.png} \\
\vspace*{1in}
\textbf{Memoria de la práctica 2} \\
Animación por Ordenador \\
\vspace*{0.5in}}
\author{Andrés Merlo Trujillo \\
andresmerlo@correo.ugr.es \\
77147239H \\ 
\vspace*{0.5in} \\
E.T.S. de Ingenierías Informática y de Telecomunicación \\
\textbf{Universidad de Granada}} \date{\today}

\hypersetup{
    colorlinks=true,
    linkcolor=black,
    citecolor=black
}

\renewcommand\maketitlehooka{\null\mbox{}\vfill}
\renewcommand\maketitlehookd{\vfill\null}

\begin{document}
\begin{titlingpage}
\maketitle
\end{titlingpage}

\tableofcontents

\newpage

\pagestyle{fancy}   %a partir de comienza el header (se salta el indice y portada)
\fancyhead[L]{Andrés Merlo Trujillo}
\fancyhead[R]{Animación por Ordenador}
%\section{Ejercicio 1}
%\begin{figure}[H]
%    \centering
%    \includegraphics[width=\textwidth]{imagenes/passwdfile.png}
%    \vspace{10pt}
%    \footnotesize{Fuente: https://...}
%\end{figure}

\section{Ejercicio 1 - Pelota rodando}

En este ejercicio se pide implementar la animacion cuatro pelotas (o un haciendo un cuadrado) que parten desde un punto A y llegan al mismo tiempo a otro punto B haciendo, mediante curvas, animaciones distintas.

\bigskip

Cabe destacar que los \textit{keyframes} para todas las pelotas son exactamente los mismos, pero cambiando la forma de la curva. Entonces, los \textit{keyframes} son los siguientes:

\begin{enumerate}
    \item Instante 0: Posición inicial en el punto A.
    \item Instante 50: Posición final en el punto B.
\end{enumerate}

Y de manera gráfica, los \textit{keyframes} son:

%fotos de una bola nada mas con los keyframes

\bigskip

%foto con todas las pelotas xd

Estas animaciones las voy a dividir en subapartados para explicar mejor como son las curvas:

\subsection{Movimiento uniforme}

Para esta animacion es necesario usar un curva lineal, para que en cada instante de tiempo se avance siempre la misma distancia. La curva para esta pelota tiene la siguiente forma:

%foto de la curva

En la animación final se puede apreciar como mantiene la velocidad desde el primer momento hasta que acaba su recorrido.

\subsection{Aceleración}

Para realizar esta animación se debe usar una curva exponencial, cuya pendiente sea creciente. Esto va a hacer que progresivamente vaya recorriendo la pelota más distancia para la misma cantidad de tiempo. 

La curva tiene la siguiente forma:

%foto de la grafica

En la animación se puede apreciar como poco a poco la pelota va acelerando cada vez más, hasta para bruscamente en en el putno B.

Cabe decir que he modificado la forma de la curva para que se pueda apreciar mejor la aceleración en la animación resultante, ya que con la curva por defecto del programa no se puede apreciar del todo bien.

\subsection{Deceleración}

En este caso se debe usar una curva similar para la de aceleración, pero con una pendiente que vaya siendo cada vez menor. Esto resultará en que la pelota cada vez va a recorrer menos distancia para la misma cantidad de tiempo, haciendo que al final frene.

La curva tiene la siguiente forma:

%foto de la curva

En la aniamción resultante la pelota comienza moviéndose muy rápido y conforme se acerca a la posición final va frenando.

Al igual que con la aceleración, he modificado la curva para que el frenado sea más pronunciado.

\subsection{Aceleración y deceleración}

Esta animación es la combinación de la versión de aceleración y deceleración, haciendo que la pelota acelere cuando salga del punto A y vaya frenando cuando se acerque al punto B. 

La curva resultante es la unión de dichas versiones, haciendo que la pendiente sea progresivamente crenciente, hasta llegar a un punto en el que la pendiente decrece. 

Dicha curva es la siguiente:

%imagen de la curva

El resultado es que la pelota acelera al principio y cuando se acerca al punto final comienza a frenar. Ademas, como en los anteriroes subapartados, he modificado la curva para que el resultado sea mas facil de ver.


\section{Ejercicio 2 - Salto de coche}

En este ejercicio se pide animar un coche que arranca desde una posición, para luego subir una rampa y saltar por los aires, para finalmente cuando toca el suelo frenar.

% no se si poner como he hecho la rampa.

%reescribir esto
Cabe destacar que para animar la rotacion para subir a la rampa, he cambiado el pivote para poner justo en el eje de las ruedas traseras, ya que es como realmente se deberia mover al intentar subir dicha rampa. 

Además, la jerarquía utilizada ha sido la de dejar como padre el cubo más grande del vehículo y como hijos las ruedas y el cubo más pequeño que hace de pilares y cristales.


La animación consiste en 8 \textit{keyframes}, que son los siguientes:

\begin{enumerate}
    \item Instante 0: El vehículo se encuentra en la posición inicial.
    \item Instante 13: El vehículo ha realizado la aceleración.
    \item Instante 26: El vehículo se encuentra en el instante anterior de subir la rampa.
    \item Instante 28: El vehículo ha subido la rampa y se encuentra al principio de la misma.
    \item Instante 32: El vehículo se encuentra en el instante anterior de estar en el aire; es decir, se encuentra tocando el final de la rampa solo con la rueda trasera.
    \item Instante 44: El vehículo se encuentra en el punto más alto del lanzamiento y ya recto.
    \item Instante 56: El vehículo ha tocado el suelo después del salto.
    \item Instante 85: Finalmente, el vehículo frena hasta pararse del todo.
\end{enumerate}

A modo visual, los \textit{keyframes} son los siguientes:

%fotos de los keyframes

Y las curvas utilizadas para la animación:

%fotos de las curvas, separadas por partes (si no son muchas)

% reescribir lo de abajo con chatgpt
Como se puede ver en la curva roja (posición en el eje X), al principio la curva tiene forma exponencial creciente, para simula la aceleración. A continuación, todo el trayecto se mantiene lineal, hasta llegar a la parte de frenado, donde tiene forma exponencial para que frene y se pare.

En cuanto a la curva de color azul (Posición en el eje Z), solo es usada para animar la subida de la rampa y la trayectoria que sigue el vehículo cuando sale despedido de la rampa. Entonces, la parte en la que sube a la rampa es lineal, con la misma pendiente que la rampa. Mientras que la animación de la trayectoria se hace con dos funciones exponenciales, para animar la fuerza que realiza la gravedad en la parte de la subida y para animar la aceleración de la gravedad en la bajada. Cabe destacar que, en este caso, la forma que tenga la curva será reflejada en la animación directamente; es decir, si la trayectoria se hubiera hecho lineal, el lanzamiento hubiese tenido forma triangular.

En cuanto a la curva verde (Rotación en el eje Y), es usada para rotar el vehículo en el cambio de pendiente de la subida de la rampa y cuando se endereza durante el lanzamiento del vehículo.
% fin rescritura

\section{Ejercicio 3 - Pelota botando}

En este ejercicio se pide animar una pelota que rebota varias veces sobre una mesa y que finalmente se cae al suelo. 

La mesa la he realizado usando 5 cubos: uno grande para el tablero y otros 4 del mismo tamaño para hacer las patas. Además, la jerarquía utilizada ha sido de dejar como padre al tablero y como hijos las patas.

%rescribir
En cuanto a la animacion, he utilizado un factor de 2 para la altura del rebote y para la distancia recorrida entre rebotes; es decir, que en cada salto su altura y distancia de divide a la mitad. Esta aniamción la he realizado usando 10 \textit{keyframes}, que son:

\begin{enumerate}
    \item 
\end{enumerate}

\end{document}
