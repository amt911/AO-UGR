\documentclass[]{article}
\usepackage[utf8]{inputenc}
\usepackage[spanish]{babel}
\usepackage{graphicx, graphics, float, fancyhdr, titling, caption, subcaption, amsmath}
\usepackage{listings,xcolor}
\usepackage[a4paper, total={6in, 9.5in}]{geometry}
\usepackage{fancyhdr}
\usepackage{hyperref}   %para que funcione addcontentsline debe ser la ultima que se cargue

\usepackage{blindtext}
\usepackage{mwe}

%\setcounter{secnumdepth}{-2}       %Poner solo esto si no se quieren numero delante de las secciones y niveles inferiores.

\renewcommand{\footrulewidth}{0.4pt}
\title{
\includegraphics[width=1.75in]{imagenes/UGR-Logo.png} \\
\vspace*{1in}
\textbf{Memoria de la Práctica 6} \\
Animación por Ordenador \\
\vspace*{0.5in}}
\author{Andrés Merlo Trujillo \\
andresmerlo@correo.ugr.es \\
77147239H \\ 
\vspace*{0.5in} \\
E.T.S. de Ingenierías Informática y de Telecomunicación \\
\textbf{Universidad de Granada}} \date{\today}

\hypersetup{
    colorlinks=true,
    linkcolor=black,
    citecolor=black
}

\renewcommand\maketitlehooka{\null\mbox{}\vfill}
\renewcommand\maketitlehookd{\vfill\null}

\definecolor{codegreen}{rgb}{0,0.6,0}
\definecolor{codegray}{rgb}{0.5,0.5,0.5}
\definecolor{codepurple}{rgb}{0.58,0,0.82}
\definecolor{backcolour}{rgb}{0.95,0.95,0.92}

\lstdefinestyle{mystyle}{
    backgroundcolor=\color{backcolour},   
    commentstyle=\color{codegreen},
    keywordstyle=\color{magenta},
    numberstyle=\tiny\color{codegray},
    stringstyle=\color{codepurple},
    basicstyle=\ttfamily\footnotesize,
    breakatwhitespace=false,         
    breaklines=true,                 
    captionpos=b,                    
    keepspaces=true,                 
    numbers=left,                    
    numbersep=5pt,                  
    showspaces=false,                
    showstringspaces=false,
    showtabs=false,                  
    tabsize=2,
    literate=
  {á}{{\'a}}1 {é}{{\'e}}1 {í}{{\'i}}1 {ó}{{\'o}}1 {ú}{{\'u}}1
  {Á}{{\'A}}1 {É}{{\'E}}1 {Í}{{\'I}}1 {Ó}{{\'O}}1 {Ú}{{\'U}}1
  {à}{{\`a}}1 {è}{{\`e}}1 {ì}{{\`i}}1 {ò}{{\`o}}1 {ù}{{\`u}}1
  {À}{{\`A}}1 {È}{{\`E}}1 {Ì}{{\`I}}1 {Ò}{{\`O}}1 {Ù}{{\`U}}1
  {ä}{{\"a}}1 {ë}{{\"e}}1 {ï}{{\"i}}1 {ö}{{\"o}}1 {ü}{{\"u}}1
  {Ä}{{\"A}}1 {Ë}{{\"E}}1 {Ï}{{\"I}}1 {Ö}{{\"O}}1 {Ü}{{\"U}}1
  {â}{{\^a}}1 {ê}{{\^e}}1 {î}{{\^i}}1 {ô}{{\^o}}1 {û}{{\^u}}1
  {Â}{{\^A}}1 {Ê}{{\^E}}1 {Î}{{\^I}}1 {Ô}{{\^O}}1 {Û}{{\^U}}1
  {ã}{{\~a}}1 {ẽ}{{\~e}}1 {ĩ}{{\~i}}1 {õ}{{\~o}}1 {ũ}{{\~u}}1
  {Ã}{{\~A}}1 {Ẽ}{{\~E}}1 {Ĩ}{{\~I}}1 {Õ}{{\~O}}1 {Ũ}{{\~U}}1
  {œ}{{\oe}}1 {Œ}{{\OE}}1 {æ}{{\ae}}1 {Æ}{{\AE}}1 {ß}{{\ss}}1
  {ű}{{\H{u}}}1 {Ű}{{\H{U}}}1 {ő}{{\H{o}}}1 {Ő}{{\H{O}}}1
  {ç}{{\c c}}1 {Ç}{{\c C}}1 {ø}{{\o}}1 {Ø}{{\O}}1 {å}{{\r a}}1 {Å}{{\r A}}1
  {€}{{\euro}}1 {£}{{\pounds}}1 {«}{{\guillemotleft}}1
  {»}{{\guillemotright}}1 {ñ}{{\~n}}1 {Ñ}{{\~N}}1 {¿}{{?`}}1 {¡}{{!`}}1,
  extendedchars=true
}

\lstset{style=mystyle}

\lstdefinelanguage{MaxScript}{
  keywords={break, case, catch, collect, continue, coordsys, default, do, else, exit, false, for, fn, global, if, in, local, macroScript, not, of, on, plugin, return, rollouts, silent, struct, then, to, true, try, undo, utilities, when, while, quat, rotate, move, normalize, distance, ray, intersectRay},
  keywordstyle=\color{blue}\bfseries,
%   ndkeywords={!=, #, #_, ##, %, &amp;, \&, \&, *, **, +, -, /, //, :, &lt;&lt;, &gt;&gt;, &lt;=, &gt;=, ==, ^, ~, ~=, +=, -=, *=, /=, //=, ^=, &=, &lt;&lt;=, &gt;&gt;=},
%   ndkeywordstyle=\color{red}\bfseries,
  identifierstyle=\color{black},
  sensitive=true,
  comment=[l]{--},
  morecomment=[s]{/*}{*/},
  commentstyle=\color{codegreen}\ttfamily,
  stringstyle=\color{purple}\ttfamily,
  morestring=[b]',
  morestring=[b]"
}

\begin{document}
\begin{titlingpage}
\maketitle
\end{titlingpage}

\tableofcontents

\newpage

\pagestyle{fancy}   %a partir de comienza el header (se salta el indice y portada)
\fancyhead[L]{Andrés Merlo Trujillo}
\fancyhead[R]{Animación por Ordenador}
%\section{Ejercicio 1}
%\begin{figure}[H]
%    \centering
%    \includegraphics[width=\textwidth]{imagenes/passwdfile.png}
%\end{figure}


\section{Introducción}

En esta práctica se pide implementar una interfaz de usuario utilizando el lenguaje MAXScript, con el objetivo de poder realizar una animación proceduralmente. Más concretamente un objeto que salte sobre un conjunto de plataformas y tenga a su vez algún modificador.
\section{Traslación}

Para realizar la traslación de la pelota, es necesario utilizar el vector Up de la pelota guía y multiplicarlo poor -1, de forma que el vector resultante mire hacia abajo. Una vez realizado esto, se debe lanzar un rayo para intersecar la superficie, así se obtiene un vector con la posición en la que ha intersecado.

\bigskip

Finalmente, se modifica el valor \textit{pos} de la pelota para que sea igual que al punto intersecado. Al realizar esto la pelota aparecerá hundida en el suelo, para que se posicione bien se debe calcular el vector que va desde la pelota hasta la pelota guía, normalizándolo y multiplicándolo por el radio de la pelota. Con este vector, se puede utilizar el comando \verb|move| una vez para levantar la pelota, ya que al tener de módulo el radio, va a moverse lo necesario.

% foto de hundido y despues.


La función para implementar esta funcionalidad es el siguiente:

% codigo
\lstinputlisting[language=MaxScript,firstline=1,lastline=28 ]{../eje_MerloTrujilloAndres_AO_P6.ms}


Y a continuación hay capturas con la pelota en distintos lugares y posicionada correctamente:

% dos o tres fotos de la pelota a distintas alturas.
\section{Rotación}

Para realizar la rotación, es necesario saber la posición en el instante anterior de la pelota guía, y la posición en el instante actual, para así obtener el vector con la dirección. Cabe destacar que es necesario normalizarlo para que el paso siguiente funcione bien.

\bigskip

Una vez obtenido este vector y haciendo uso del vector Up de la pelota guía, se hace el producto vectorial para así obtener el vector left (o right, no importa), que es el que se usará para girar la pelota.

\bigskip

Por último, para saber cuantos grados debe rotar la pelota, es necesario calcular la distancia del vector primero sin normalizar, de forma que así se obtenga la distancia que ha recorrido. Después, se debe utilizar la fórmula de la longitud del arco de circunferencia, poniendo como incógnita el ángulo. Se debe hacer así porque hay que convertir la distancia recta en distancia circular, para calcular los ángulos. 

\bigskip

La fórmula de la longitud de la circunferencia, cuya incógnita es el ángulo, es: $\alpha = \frac{360 \cdot L}{2 \cdot \pi \cdot r} $, donde L es la longitud recorrida y r el radio de la pelota.

\bigskip

El código para realizar esto es:

% codigo [firstline=300,lastline=500]
\lstinputlisting[language=MaxScript,firstline=31,lastline=56]{../eje_MerloTrujilloAndres_AO_P6.ms}

Cabe destacar que esta función no se debe ejecutar en el primer instante, al necesitar tener el valor de posición de la pelota guía en el instante anterior, por lo que hay que esperar al segundo instante para que empiece a funcionar.

\bigskip

A continuación muestro algunas imágenes de la rotación:

% dos o tres fotos en los mismos instantes que antes de la traslacion

% \bibliographystyle{plainurl} % We choose the "plain" reference style
% \bibliography{bib} % Entries are in the refs.bib file

\end{document}
