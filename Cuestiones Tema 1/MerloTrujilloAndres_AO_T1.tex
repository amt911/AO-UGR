\documentclass{article}
\usepackage[utf8]{inputenc}
\usepackage[spanish]{babel}
\usepackage{graphicx, graphics, float, fancyhdr, titling}
\usepackage{listings}
\usepackage[a4paper, total={6in, 9.5in}]{geometry}
\usepackage{fancyhdr}
\usepackage{hyperref}   %para que funcione addcontentsline debe ser la ultima que se cargue

%\setcounter{secnumdepth}{-2}       %Poner solo esto si no se quieren numero delante de las secciones y niveles inferiores.

\renewcommand{\footrulewidth}{0.4pt}
\title{
\includegraphics[width=1.75in]{imagenes/UGR-Logo.png} \\
\vspace*{1in}
\textbf{Cuestiones Tema 1} \\
Animación por Ordenador \\
\vspace*{0.5in}}
\author{Andrés Merlo Trujillo \\
andresmerlo@correo.ugr.es \\
77147239H \\ 
\vspace*{0.5in} \\
E.T.S. de Ingenierías Informática y de Telecomunicación \\
\textbf{Universidad de Granada}} \date{\today}

\hypersetup{
    colorlinks=true,
    linkcolor=black,
}

\renewcommand\maketitlehooka{\null\mbox{}\vfill}
\renewcommand\maketitlehookd{\vfill\null}

\begin{document}
\begin{titlingpage}
\maketitle
\end{titlingpage}

\tableofcontents

\newpage

\pagestyle{fancy}   %a partir de comienza el header (se salta el indice y portada)
\fancyhead[L]{Andrés Merlo Trujillo}
\fancyhead[R]{Animación por Ordenador}
%\section{Ejercicio 1}
%\begin{figure}[H]
%    \centering
%    \includegraphics[width=\textwidth]{imagenes/passwdfile.png}
%\end{figure}

\section{¿Qué es la animación?}

La animación es un proceso que permite dar la ilusión de movimiento a las imágenes, haciendo que se dé la ilusión de que la escena está cambiando. [WIKIPEDIA] 

\bigskip

Existen diversas técnicas de animación, pero todas ellas residen en el hecho de realizar múltiples imágenes con cambios ligeros y que se presentan de manera sucesiva de manera rápida y constante, dando la sensación de movimiento y que las cosas cambian (luz, forma, color, posición, etc.). Además, por norma general, se suelen usar 24 fotogramas por segundo para generar las animaciones (aunque también puede ser más alta, generando una sensación de movimiento más fluida). [TRANS]

\bigskip

A lo largo de la historia han existido diversos precursores de la animación, desde pinturas prehistóricas hasta proyecciones de imágenes. [TRANS]

\section{Buscad información sobre el fenómeno Phi e indicad la relación con la animación y lo comentado hasta ahora.}

El fenómeno phi, es una ilusión óptica que permite tener sensación de movimiento en donde hay imágenes estáticas que cambian a una frecuencia determinada [WIKIPEDIA2]. 

\bigskip

Todo esto se basa en la persistencia retiniana (o persistencia visual), que consiste en una limitación del ojo humano para procesar las imágenes a tanta velocidad, haciendo que se vea movimiento en esta sucesión [WIKIPEDIA3]. Esto hace que el cerebro interpole las imágenes para darles suavidad.

\bigskip

La relación con la animación se basa en estos dos conceptos, ya que sirven de ``herramienta'' básica para poder generar la sensación de movimiento esperado de las imágenes estáticas creadas para hacer la animación. 


\section{Busca información sobre antiguos artefactos que aplicaban la idea de persistencia visual}

A continuacion voy a pasar a explicar como funcionaba cada artefacto mostrado en las transparencias.

\begin{itemize}
    \item \textbf{Linterna mágica: }Aparato optico que consistia en un juego de lentes, con una camara interna donde se colocaba la fuente de luz, y delante se colocaban las imagenes, que eran iluminadas y proytectadas a los espectadores. [WIKIPEDIA4] 
    
    Estas imagenes se pueden clasificar en distintos tipos, pero las que son utiles para la asignatura son las mecaniacas. Por ejemplo, las \textit{Simple Slipping Slides} consistian en dos placas que mostraban dos momentos distintos y para hacer simular movimiento se cambiaba rapidamente a la otra placa.[WIKIPEDIA4]

    Otro sistema era el de uso de un disco con varias imagenes, que iba rotando y presentando las distintas imagenes para dar sensacion de movimeinto.

    Esto tiene que ver con la persistencia visual en el hecho de que se presentan dos imagenes los suficientemente rapido como para dar sensacion de movimiento; es decir, su funcionamento se basa en este fenómeno si se quiere crear movimeinto en las escenas.

    \item \textbf{Taumátropo: }Es un juguete consistente en un disco con dos imagenes distintas a ambos lado y una cuerda a cada lado del disco. Haciendo girar rapidamente el taumátropo, se genera la ilusión de que ambas imágenes están juntas. [WIKIPEDIA5]
    
    Los dibujos mas comunes son un pajaro y una jaula, que al hacerlos girar da la sensacion de que el pajaro esta en dicha jaula.

    Este juguete tiene relacion con la persistencia retiniana al mostrar una secuencia de imagenes rapidamente (en el caso de este juguete son dos imagenes), haciendo que se produzca dicho fenomeno y produciendo asi que al seguir estando la imagen anterior en la retina, se fusionen ambas imagenes.

    \item \textbf{Kinetoscopio: }Precursos moderno del proyector de peliculas, al hacer uso de una cinta para almacenar las iamgenes. Sin embargo, no se podia proyectar en una pantalla, era solo para uso individual.[WIKIPEDIA6]
    
    Consistia en una caja en la que dentro se instalaba la cinta con todas las imagenes y de una fuente de luz con un obturador de luz de alta velocidad.[WIKIPEDIA6]

    Esta relacionado con la persitencia visual al pasar rapidamente las imagenes de la cinta por el obturador, y dando la sensacion de movimiento para aquella persona que este observando por la lente.

    \item \textbf{Zoótropo: }Máquina estroboscopica, compuesta por un tambor circular con cortes y cuyo interior tiene imagenes secuenciales del movimeinto de un objeto o escena. El observador tiene que mirar por las ranuras para poder observar las imagenes y poder apreciar el movimeinto. Ademas, tinee la ventaja de que varias personas pueden usarlo a la vez.[WIKIPEDIA7]
    
    Tiene relacion con la presistencia visual al mostrar las imagenes de manera rapida y continuada. Esto se consigue mediante las ranuras, que actuan de visor en el momento en el que se encuentran delante del espectador y cuando no esten delante desaparezca la imagen para que la siguiente ranura muestre la siguiente. Esto al final hace el efecto de que aparezca y desaparezca la imagen.
\end{itemize}


\section{Busca informacion sobre estas cuatro tecnicas centrandote principalmente en las dos ultimas}

Voy a dividir cada apartado en subsecciones.

\subsection{Rotoscopia}

La rotoscopia es una tecnica de animcion que los animadores utilizan consistente en dibujar las imagenes que componen una animacion fotograma a fotograma, utilizando como base una grabacion real de la animacion que se desea dibujar. Antiguamente, se hacia uso del rotoscopio, que es una mquina que proyectaba la grabacion en un panel de cristal y donde el dibujante calcaba el fotograma a dibujo. [https://en.wikipedia.org/wiki/Rotoscoping]

\bigskip

Algunos ejemplos de peliculas son \textit{Blancanieves y los siete enanitos}, \textit{Alicia en el pais de las maravillas}. [https://soydecine.com/rotoscopia/] y \textit{Star Wars} para crear los sables laser [https://en.wikipedia.org/wiki/Rotoscoping]

\subsection{Animacion paso a paso}

Tecnica de animacion consistente en la elaboracion de imagenes individuales (cuadros) cuyo movimiento o cualquiera de los demas atributos varia ligeramente. Al mostrarse de manera secuencial se crea la ilusion de movimeinto. Ha sido ampliamente utilizada para dibujos anuimadios y programas de teelvision. Hoy en dia se suele realizar este tipo de animacion mediante el ordendador.

\bigskip

Un ejemplo de animacion paso a paso es el \textit{Stop Motion}, en el que se fotografian escenas y personajes, y en cada fotograma se varia un poco para dar la sensacion de movimiento al unirlas.


\subsection{Animacion por claves}

Tecnica de aniamcion utilizada para la animacion digital y peliculas. Consiste en definir puntos clave de una animacion, indicando en dicho punto la posicion, rotacion, escala u otras propiedades. Despues, mediante un software de animacion (Blender, 3ds Max, etc) se interpolan las posiciones intermedias entre estos puntos clave, creando una animacion fluida. [https://www.adobe.com/creativecloud/video/discover/keyframing.html]

\bigskip

Ademas, se puede modificar la forma en la que se interpola dicha animacion mediante funciones, haciendo que, por ejemplo, un objeto acelere hasta el siguiente punto clave, mantenga la velocidad, acelere y frene, vaya frenando o incluso modelar una funcion personalizada.

\bigskip

Por ultimo, cabe destacar que este tipo de animacion permite ahorrar al animador mucho tiempo, al no tener que animar fotograma a fotograma (animacion paso a paso) los objetos. [https://www.adobe.com/creativecloud/video/discover/keyframing.html]


\subsection{Animacion procedural}

Es una tecnica de animacion usada para generar automaticamente animaciones en tiempo real, permitiendo tener una serie de animaciones mucho mas amplia que con las tecnicas anteoriores. %[https://en.wikipedia.org/wiki/Procedural_animation]

\bigskip

Normalmente es usado para generar sistemas de particulas (humo, agua, fuego, etc), para la animacion de la ropa de un personaje o el pelo del mismo. Tambien se puede usar para dar realismo al mundo, al poder usarse para hacer sistemas de física ragdoll (en inglés \textit{Ragdoll Physics}) [https://es.wikipedia.org/wiki/F%C3%ADsica_ragdoll]. 
Esto hace que las animaciones de muerte, por ejemplo, sean dinamicas y mas realistas dependiendo de la situacion en la que se encuetnre el personaje, frente a la sensacion de rigidez que puede dar las otras tecnicas de animacion, en las que dicha animacion estaria predefinida.

\bigskip

Como desventaja se puede decir que tiene un mayor consumo de la CPU frente a las animaciones manuales, al tener que calcular el siguiente fotograma de la animacion en tiempo real.
\end{document}
