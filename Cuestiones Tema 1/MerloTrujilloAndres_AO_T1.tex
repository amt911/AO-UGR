\documentclass{article}
\usepackage[utf8]{inputenc}
\usepackage[spanish]{babel}
\usepackage{graphicx, graphics, float, fancyhdr, titling}
\usepackage{listings}
\usepackage[a4paper, total={6in, 9.5in}]{geometry}
\usepackage{fancyhdr}
\usepackage{hyperref}   %para que funcione addcontentsline debe ser la ultima que se cargue

%\setcounter{secnumdepth}{-2}       %Poner solo esto si no se quieren numero delante de las secciones y niveles inferiores.

\renewcommand{\footrulewidth}{0.4pt}
\title{
\includegraphics[width=1.75in]{imagenes/UGR-Logo.png} \\
\vspace*{1in}
\textbf{Cuestiones Tema 1} \\
Animación por Ordenador \\
\vspace*{0.5in}}
\author{Andrés Merlo Trujillo \\
andresmerlo@correo.ugr.es \\
77147239H \\ 
\vspace*{0.5in} \\
E.T.S. de Ingenierías Informática y de Telecomunicación \\
\textbf{Universidad de Granada}} \date{\today}

\hypersetup{
    colorlinks=true,
    linkcolor=black,
}

\renewcommand\maketitlehooka{\null\mbox{}\vfill}
\renewcommand\maketitlehookd{\vfill\null}

\begin{document}
\begin{titlingpage}
\maketitle
\end{titlingpage}

\tableofcontents

\newpage

\pagestyle{fancy}   %a partir de comienza el header (se salta el indice y portada)
\fancyhead[L]{Andrés Merlo Trujillo}
\fancyhead[R]{Animación por Ordenador}
%\section{Ejercicio 1}
%\begin{figure}[H]
%    \centering
%    \includegraphics[width=\textwidth]{imagenes/passwdfile.png}
%\end{figure}

\section{¿Qué es la animación?}

La animación es un proceso que permite dar la ilusión de movimiento a las imágenes, haciendo que se dé la ilusión de que la escena está cambiando. [WIKIPEDIA] 

\bigskip

Existen diversas técnicas de animación, pero todas ellas residen en el hecho de realizar múltiples imágenes con cambios ligeros y que se presentan de manera sucesiva de manera rápida y constante, dando la sensación de movimiento y que las cosas cambian (luz, forma, color, posición, etc.). Además, por norma general, se suelen usar 24 fotogramas por segundo para generar las animaciones (aunque también puede ser más alta, generando una sensación de movimiento más fluida). [TRANS]

\bigskip

A lo largo de la historia han existido diversos precursores de la animación, desde pinturas prehistóricas hasta proyecciones de imágenes. [TRANS]

\section{Buscad información sobre el fenómeno Phi e indicad la relación con la animación y lo comentado hasta ahora.}

El fenómeno phi, es una ilusión óptica que permite tener sensación de movimiento en donde hay imágenes estáticas que cambian a una frecuencia determinada [WIKIPEDIA2]. 

\bigskip

Todo esto se basa en la persistencia retiniana (o persistencia visual), que consiste en una limitación del ojo humano para procesar las imágenes a tanta velocidad, haciendo que se vea movimiento en esta sucesión [WIKIPEDIA3]. Esto hace que el cerebro interpole las imágenes para darles suavidad.

\bigskip

La relación con la animación se basa en estos dos conceptos, ya que sirven de ``herramienta'' básica para poder generar la sensación de movimiento esperado de las imágenes estáticas creadas para hacer la animación. 


\section{Busca información sobre antiguos artefactos que aplicaban la idea de persistencia visual}

A continuacion voy a pasar a explicar como funcionaba cada artefacto mostrado en las transparencias.

\begin{itemize}
    \item \textbf{Linterna mágica: }Aparato optico que consistia en un juego de lentes, con una camara interna donde se colocaba la fuente de luz, y delante se colocaban las imagenes, que eran iluminadas y proytectadas a los espectadores. [WIKIPEDIA4] 
    
    Estas imagenes se pueden clasificar en distintos tipos, pero las que son utiles para la asignatura son las mecaniacas. Por ejemplo, las \textit{Simple Slipping Slides} consistian en dos placas que mostraban dos momentos distintos y para hacer simular movimiento se cambiaba rapidamente a la otra placa.[WIKIPEDIA4]

    Otro sistema era el de uso de un disco con varias imagenes, que iba rotando y presentando las distintas imagenes para dar sensacion de movimeinto.

    Esto tiene que ver con la persistencia visual en el hecho de que se presentan dos imagenes los suficientemente rapido como para dar sensacion de movimiento; es decir, su funcionamento se basa en este fenómeno si se quiere crear movimeinto en las escenas.

    \item \textbf{Taumátropo: }Es un juguete consistente en un disco con dos imagenes distintas a ambos lado y una cuerda a cada lado del disco. Haciendo girar rapidamente el taumátropo, se genera la ilusión de que ambas imágenes están juntas. [WIKIPEDIA5]
    
    Los dibujos mas comunes son un pajaro y una jaula, que al hacerlos girar da la sensacion de que el pajaro esta en dicha jaula.

    Este juguete tiene relacion con la persistencia retiniana al mostrar una secuencia de imagenes rapidamente (en el caso de este juguete son dos imagenes), haciendo que se produzca dicho fenomeno y produciendo asi que al seguir estando la imagen anterior en la retina, se fusionen ambas imagenes.

    \item 
\end{itemize}


\end{document}
